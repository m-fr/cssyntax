\input opmac
\input cssyntax
\input mfrdoc/mfrdoc
\input opmac-bib

\def\priklad{\numberedpar A{Příklad}}
\addto\tthook{\ttline=0}
\def\hisyntaxX#1\egroup{\let\n=\relax \let\NLh=\relax \let\U=\relax
\let\TABchar=\relax % for OPmac trick 0151
\adef{ }{\n\ \n}\adef\^^M{\n\NLh\n}\edef\tmpb{#1\egroup}
\replacestrings{\n\egroup}{\egroup}%
\let\NLh=\par \def\n{}\def\U##1{\csname##1\endcsname}%
\tentt\localcolor\tmpb\egroup}

\title {Zvýraznění syntaxe C\#}
\subtitle {\TeX (\OPmac)}
\author {Michal Franc}
\date {Květen 2019}
\worklanguage[CZ]
\makefront

\chap Makro CSSyntax

Toto je referenční dokument k~makru CSSyntax.
Makro slouží k~zvýraznění syntaxe jazyka C\# v~\TeX ových dokumentech používajících \OPmac\cite[opmac] a vychází z~\OPmac\ triků\cite[opmact] 124 a 125.
Toto makro vzniklo v~rámci semestrální práce v~předmětu Typografie a \TeX\ na FIT ČVUT.

\sec Zavedení makra

Celé makro lze vložit přímo do dokumentu.
Pokud nechcete vkládat makro přímo do dokumentu, můžete připojit soubor "cssyntax.tex".
Toho docílíte pomocí tohoto příkazu:
\begtt
\input cssyntax
\endtt
Aby připojení makra proběhlo hladce, je samozřejmě nutné umístit "cssyntax.tex" do některého z~adresářů prohledávaných při sestavování dokumentu.

Po připojení makra lze před verbatim prostředí přidat "\hisyntax{CS}" nebo "\ghisyntax{CS}" pro lokální nebo globální nastavení obarvení.
Začátek verbatim prostředí může ve zdrojovém textu vypadat například takto:
\begtt
\hisyntax{CS}
\begtt
string foo = "bar";
...
\endtt

\sec Omezení makra

Pokud jsou některé speciální znaky použity v~jiném makru, můžou vzniknout potíže při vyhledávání a nahrazování skupin znaků.
Tyto skupiny se v~takovém případě neobarví korektně.
Nejčastěji se tak stává u~uvozovek používaných v~příkazu "\activettchar" pro nastavení řádkového verbatim prostředí.
Tomu lze snadno zabránit předřazením makra pro obarvování syntaxe před ostatními makry.

\sec Další zlepšení

Makro vyhovujícím způsobem obarví téměř libovolný kód.
Avšak stále skýtá prostor pro další zlepšení.
Přidat lze například tyto věci:
\begitems
* obarvení proměnných a metod;
* obarvení kontextuálních slov na základě kontextu;
* obarvení prefixů pro řetězce;
* obarvení proměnných v~interpolovaných řetězcích.
\enditems

\chap Elementy jazyka C\#

V~této kapitole je probírána syntaxe jazyka C\# a její zvýraznění pomocí makra CSSyntax.
Syntaktické elementy a příklady jejich použití jsou čerpány z~referenční příručky jazyka\cite[csharp].

\sec Komentáře

C\# podporuje blokové i řádkové komentáře.
Řádkové komentáře jsou uvozeny pomocí dvou lomítek "//" a pokračují až do konce řádku.
Blokové komentáře jsou uzavřeny mezi "/*" a "*/".
Makro také podporuje obarvení dokumentačních komentářů.
Ty jsou uvozeny pomocí tří lomítek "///" a obsahují xml tagy.

\ghisyntax{CS}
\begtt
int x;  // Toto je řádkový komentář

/* Toto je blokový komentář */
\endtt
\priklad Komentáře v~C\#

\sec Klíčová slova

Klíčová slova jsou vyhrazené identifikátory mající speciální význam v~rámci jazyka, například:
\begtt
class      continue   explicit   foreach    typeof
\endtt

C\# dále definuje kontextuální klíčová slova.
Tato slova mají speciální význam v~určitém kontextu, ale nejsou rezervovanými slovy jazyka.
Mezi kontextuální slova se mimo jiné řadí:
\begtt
alias      dynamic    group      select     yield
\endtt

V~současné verzi jsou kontextuální i klíčová slova zpracovávána stejným způsobem.
To způsobí obarvení kontextuálního klíčového slova i mimo kontext, tedy v~momentě, kdy není považováno za klíčové.

Celý seznam klíčových slov je v~příloze~\ref[keywords].

\sec Literály

Mezi literály lze obecně zařadit vše, co označuje konkrétní hodnotu.
Pro účely zvýraznění se tato skupina dále dělí na řetězce, čísla atd.
Následující literály zůstanou bez zařazení do podskupin:

\begtt
true       false      null
\endtt

Tyto literály jsou zároveň klíčovými slovy, a proto jsou zpracovávány podobným způsobem.

\secc Řetězce

Řetězce jsou uzavřeny mezi dvojici uvozovek.
Jazyk dále definuje prefixy "@" pro verbatim prostředí a "$" pro interpolaci řetězců.
Prefixy momentálně nepodléhají obarvení.

\begtt
string s1 = "hello \t world";
string s2 = @"hello \t world";
string s3 = $" He said: \"{s1}\"";
string s3 = $@" He said: ""{s1}""";
\endtt
\priklad Řetězce

\secc Znaky

Znakové literály se uvozují pomocí dvojice jednoduchých uvozovek.
Samotný literál může být jeden znak nebo escape sekvence.

\begtt
char[] vowels = {'a', 'e', 'i', 'o', 'u'};
char[] newline = {'\r', '\n', '\u0085', '\u2028)', '\u2029'};
\endtt
\priklad Znaky

\secc Čísla

V~C\# se vyskytují celá a desetinná čísla.
Celá čísla můžeme zapsat v~desítkové nebo hexadecimální soustavě a případně k~nim připojit suffix určující přesný typ.
V~případně desetinných čísel se literály zapisují ve tvaru:

"<celá část>.<desetinná část>e<exponent><suffix>".

\begtt
    10         13ul         -15L       0xff         0x56u
    1f         1.5d        1e10m    123.456        12.86M
\endtt
\priklad Numerické literály

\sec Direktivy preprocesoru

Direktivy preprocesoru jsou složeny z~příkazu a většinou jedné hodnoty.
Některé direktivy mají hodnot více, jiné nemají žádnou.
Zápis direktivy vypadá následovně:

"#<direktiva> <hodnota direktivy>".

\begtt
#if DEBUG
#warning DEBUG mode active
#endif
\endtt
\priklad Direktivy procesoru

\sec Atributy

Atributy umožňují programátorovi přidat k~různým entitám programu další informace.
K~těmto informacím je možné přistupovat za běhu programu.
Atributy se uvádějí uvnitř hranatých závorek.

\begtt
// Attribute class definition
[AttributeUsage(AttributeTargets.Class, AllowMultiple = true)]
public class AuthorAttribute: Attribute
{ ... }

// Attribute class usage
[Author("Brian Kernighan"), Author("Dennis Ritchie")]
class Class1
{ ... }
\endtt
\priklad Atributy v~C\#

\sec Operátory

Makro také dokáže obarvit operátory, oddělovače a další speciální znaky jazyka.
Ve výchozím stavu jsou však obarvovány černě, takže zdánlivě zůstávají stejné.

\chap Styl zvýraznění

Každá skupina konstrukcí jazyka má vlastní způsob zvýraznění.
Pro snadné rozpoznání jsou styly pojmenovány "S<Element>".
Výchozí styl, který je částečně inspirován stylem použitém v~dokumentaci jazyka, je nastaven takto:
\midinsert
\hisyntax{X}
\begtt
\let\SComment=\Green      % comments
\let\SDocComment=\Grey    % documentation comments
\let\SPreprocKW=\Magenta  % preprocessor directives
\let\SPreprocVal=\Magenta % preprocessor directive values
\let\SKeyword=\Blue       % keywords
\let\SAttribute=\Cyan     % attributes
\let\SLiteral=\Cyan       % literals
\let\SString=\Red         % strings
\let\SChar=\Red           % characters
\let\SNumber=\Black       % numbers
\let\SOper=\Black         % operators
\endtt
\priklad Výchozí nastavení barev
\endinsert

\bibchap
\usebib/c (simple) mybase

\label[keywords]
\app Klíčová slova

Zde naleznete přehled všech klíčových slov jazyka C\# 8.0.

\begtt
// Klíčová slova
abstract     as           base         bool         break
byte         case         catch        char         checked
class        const        continue     decimal      default
delegate     do           double       else         enum
event        explicit     extern       false        finally
fixed        float        for          foreach      goto
if           implicit     in           int          interface
internal     is           lock         long         namespace
new          null         object       operator     out
override     params       private      protected    public
readonly     ref          return       sbyte        sealed
short        sizeof       stackalloc   static       string
struct       switch       this         throw        true
try          typeof       uint         ulong        unchecked
unsafe       ushort       using        virtual      void
volatile     while

// Kontextuální klíčová slova
add          alias        ascending    async        await
by           descending   dynamic      equals       from
get          global       group        into         join
let          nameof       on           orderby      partial
remove       select       set          value        var
when         where        yield
\endtt

\app Ukázky obarvení kódu

\sec Dokumentační komentáře

\ttline=0
\begtt
/// <summary>This method changes the point's location by
///    the given x- and y-offsets.
/// <example>For example:
/// <code>
///    Point p = new Point(3,5);
///    p.Translate(-1,3);
/// </code>
/// results in <c>p</c>'s having the value (2,8).
/// </example>
/// </summary>

public void Translate(int xor, int yor) {
    X += xor;
    Y += yor;
}
\endtt

\sec Hello World

\verbinput (-) hello.cs

\bye