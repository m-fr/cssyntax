\input opmac
\input cssyntax
\input mfrdoc/mfrdoc
\input opmac-bib

\def\priklad{\numberedpar A{Example}}
\addto\tthook{\ttline=0}
\def\hisyntaxX#1\egroup{\let\n=\relax \let\NLh=\relax \let\U=\relax
\let\TABchar=\relax % for OPmac trick 0151
\adef{ }{\n\ \n}\adef\^^M{\n\NLh\n}\edef\tmpb{#1\egroup}
\replacestrings{\n\egroup}{\egroup}%
\let\NLh=\par \def\n{}\def\U##1{\csname##1\endcsname}%
\tentt\localcolor\tmpb\egroup}

\title {C\# syntax highlight}
\subtitle {\TeX (\OPmac)}
\author {Michal Franc}
\date {May 2019}
\worklanguage[EN]
\makefront

\chap Macro CSSyntax

This is documentation of CSSyntax macro.
The macro is C\# syntax highlighter for \TeX\ documents written with \OPmac\cite[opmac].
The macro is based on \OPmac\ tricks\cite[opmact] 124 and 125.
The work was made as part of semestral work for Typography and \TeX\ course at FIT CTU.

\sec Usage

You can paste whole macro into source document.
If you don't want to do this, you can just include file "cssyntax.tex".
That can be done using this command:
\begtt
\input cssyntax
\endtt
The "cssyntax.tex" file should be somewhere in \TeX\ search path to prevent build errors.

After inclusion of the macro you can use "\hisyntax{CS}" or "\ghisyntax{CS}" before verbatim to use highlighter locally or globally.
In source document the begin of the verbatim can look like this:
\begtt
\hisyntax{CS}
\begtt
string foo = "bar";
...
\endtt

\sec Limitations

If you are using special characters in other macro, there can occur problems during searching and replacing character groups.
In that case these groups won't be highlighted correctly.
Typical example are quotation marks used by "\activettchar" for inline verbatim.
You can input highlighter before other macros to prevent this situation.

\sec Further improvements

The highlighter works correctly with almost every C\# code.
Although there are few possible improvements to do.
There can be improved for example these things:
\begitems
* variables and methods highlight;
* contextual keywords highlight based on context;
* string prefixes highlight;
* variables highlight in interpolated strings.
\enditems

\chap C\# elements

In this chapter we will talk about C\# language syntax and about highlighting of this syntax with CSSyntax.
Syntax elements and usage examples are derived from C\# language reference guide\cite[csharp].

\sec Comments

There are both block and row comments in C\#.
Row comments begin with two slashes "//" and they are active until end of the line.
Block comments are surrounded by "/*" and "*/".
The macro also supports documentation comments.
These comments begin with three slashes "///" and they include xml tags.

\ghisyntax{CS}
\begtt
int x;  // This is row comment

/* This is block comment */
\endtt
\priklad Comments in C\#

\sec Keywords

Keywords are identifiers reserved by the language. In C\# are keywords like this:
\begtt
class      continue   explicit   foreach    typeof
\endtt

C\# also defines contextual keywords.
These words have special meaning only in given context but they are not reserved by the language.
In C\# there are for example these contextual keywords:
\begtt
alias      dynamic    group      select     yield
\endtt

Both types of keywords are currently highlighted the same way.
This causes highlight of contextual keywords even if the keyword is out of context.

You can find complete list of keywords in appendix~\ref[keywords].

\sec Literals

In common sense, literal is anything that means some particular value.
The macro highlights few different groups of literals such as strings, numbers etc.
These literals do not have own group:

\begtt
true       false      null
\endtt

These literals are keywords at the same time. According this fact they are highlighted in a similar way.

\secc Strings

Strings are enclosed by quotation marks.
Strings can also be prefixed with "@" for verbatim and with "$" for string interpolation.
These prefixes aren't currently highlighted.

\begtt
string s1 = "hello \t world";
string s2 = @"hello \t world";
string s3 = $" He said: \"{s1}\"";
string s3 = $@" He said: ""{s1}""";
\endtt
\priklad Strings

\secc Characters

Strings are enclosed by single quotation marks.
The literal itself can be one character or an escape sequence.

\begtt
char[] vowels = {'a', 'e', 'i', 'o', 'u'};
char[] newline = {'\r', '\n', '\u0085', '\u2028)', '\u2029'};
\endtt
\priklad Characters

\secc Numbers

C\# basically supports integers and decimal point numbers.
Integers can be written in decimal or hexadecimal format.
It is possible to append type suffix to them.
Decimal point numbers have this format:

"<whole part>.<decimal part>e<exponent><suffix>".

\begtt
    10         13ul         -15L       0xff         0x56u
    1f         1.5d        1e10m    123.456        12.86M
\endtt
\priklad Numbers

\sec Preprocessor directives

Preprocessor directives consist from command and typically one parameter.
There are also directives with more parameters or without any parameter.
The directive can look like this:

"#<directive command> <directive value>".

\begtt
#if DEBUG
#warning DEBUG mode active
#endif
\endtt
\priklad Preprocessor directives

\sec Attributes

Attributes enables programmers to invent new kinds of declarative information.
Programmers can then attach attributes to various program entities, and retrieve attribute information in a run-time environment.
Attributes are enclosed by square brackets.

\begtt
// Attribute class definition
[AttributeUsage(AttributeTargets.Class, AllowMultiple = true)]
public class AuthorAttribute: Attribute
{ ... }

// Attribute class usage
[Author("Brian Kernighan"), Author("Dennis Ritchie")]
class Class1
{ ... }
\endtt
\priklad Attributes in C\#

\sec Operator

The macro also supports highlighting of operators, separators etc.
But these characters aren't highlighted by default.
To highlight them you can just change highlight color for this group.

\chap Highlight style

Each group has its own highlight style.
These styles have naming convention in "S<Element>" format.
Default style inspired by documentation style sets values as follows:
\midinsert
\hisyntax{X}
\begtt
\let\SComment=\Green      % comments
\let\SDocComment=\Grey    % documentation comments
\let\SPreprocKW=\Magenta  % preprocessor directives
\let\SPreprocVal=\Magenta % preprocessor directive values
\let\SKeyword=\Blue       % keywords
\let\SAttribute=\Cyan     % attributes
\let\SLiteral=\Cyan       % literals
\let\SString=\Red         % strings
\let\SChar=\Red           % characters
\let\SNumber=\Black       % numbers
\let\SOper=\Black         % operators
\endtt
\priklad Default style
\endinsert

\bibchap
\usebib/c (simple) mybase

\label[keywords]
\app Keywords

You can see full list of keywords in C\# 8.0 language in this appendix.

\begtt
// Keywords
abstract     as           base         bool         break
byte         case         catch        char         checked
class        const        continue     decimal      default
delegate     do           double       else         enum
event        explicit     extern       false        finally
fixed        float        for          foreach      goto
if           implicit     in           int          interface
internal     is           lock         long         namespace
new          null         object       operator     out
override     params       private      protected    public
readonly     ref          return       sbyte        sealed
short        sizeof       stackalloc   static       string
struct       switch       this         throw        true
try          typeof       uint         ulong        unchecked
unsafe       ushort       using        virtual      void
volatile     while

// Contextual keywords
add          alias        ascending    async        await
by           descending   dynamic      equals       from
get          global       group        into         join
let          nameof       on           orderby      partial
remove       select       set          value        var
when         where        yield
\endtt

\app Highlight examples

\sec Documentation comments

\ttline=0
\begtt
/// <summary>This method changes the point's location by
///    the given x- and y-offsets.
/// <example>For example:
/// <code>
///    Point p = new Point(3,5);
///    p.Translate(-1,3);
/// </code>
/// results in <c>p</c>'s having the value (2,8).
/// </example>
/// </summary>

public void Translate(int xor, int yor) {
    X += xor;
    Y += yor;
}
\endtt

\sec Hello World

\verbinput (-) hello.cs

\bye